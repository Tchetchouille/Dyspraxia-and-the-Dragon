\documentclass{scrartcl}
\usepackage[utf8]{inputenc}
\usepackage{soul}

\usepackage[headsepline=.5pt]{scrlayer-scrpage}
\pagestyle{scrheadings}

\ohead{Jeremias Kuehne}
\ihead{Dyspraxia et le Dragon - Features}

\title{Dyspraxia et le Dragon}
\subtitle{Features}
\author{}
\date{}


\begin{document}
	\maketitle
	\section{Essentielles}
	\begin{itemize}
		\item \st{PJ - Mouvement (gauche - droite)}
		\item \st{PJ - Saut}
		\item \st{PJ - Épée suit le joystick}
		\item \st{PJ - Dommages reçus}
		\item \st{PJ - Collisions}
		\item \st{PJ - Barre de vie ?}
		\item \st{Dragon - Barre de vie}
		\item Celestia - Mouvements précis
		\item Celestia - Saut précis
		\item Celestia - Épée stable
		\item Dyspraxia - Mouvement imprécis
		\item Dyspraxia - Sauts imprécis
		\item Dyspraxia - Épée imprécise
		\item \st{Dragon - Dommages reçus}
		\item \st{Dragon - Collision}
		\item \st{Griffes - Dommages reçus (ou pas, à voir en playtests)}
		\item \st{Griffes - Collisions}
		\item \st{Griffes - Dommages aux PJ}
		\item \st{Griffes - Déplacement}
		\item Griffes - Telegraphing
		\item \st{Dragon - Boules de feu crachables}
		\item Dragon - Telegraphing boules de feu
		\item \st{Boules de feu - Dommages aux PJ}
		\item \st{Boules de feu - Destruction par épée}
		\item \st{Dragon - IA (choix boules de feu ou griffes)}
		\item Dyspraxia - Sprites
		\item Celestia - Sprites
		\item Épée - Sprites
		\item Dragon - Sprites
		\item Griffes - Sprites
		\item Boules de feu - Sprites
	\end{itemize}
	\section{Importantes}
	\begin{itemize}
		\item Écran d'accueil
		\item Écran de victoire
		\item Scène d'intro - Fée ("dyspraxie")
		\item Scène d'intro - Écriture / Chaussures / Se cogner 
		\item Scène d'intro - Tout le monde se moque ("bête")
		\item Scène d'intro - Celestia 
		\item Scène d'intro - Kidnapping
		\item Scène intermédiaire (passage de Dyspraxia à Celestia)
		\item Personnages - Wobbly 
		\item Arène - Sprites
		\item Saut - Sons
		\item Boules de feu - Sons
		\item Griffes - Sons
		\item Épée - Sons
	\end{itemize}
	\section{Bonnes à Avoir}
	\begin{itemize}
	\item Accueil - Musique
	\item Combat - Musique
	\item Scène d'intro - Musique
	\item Scène d'intro - Se préparer à aller combattre le dragon
	\end{itemize}
	\section{Bonus}
	\begin{itemize}
		\item Queue - Dommages reçus (ou pas, à voir en playtests)
		\item Queue - Collisions
		\item Queue - Dommages aux PJ 
		\item Queue - Sprites
		\item Options - Menu
		\item Options - Accessibilité
	\end{itemize}
	
\end{document}